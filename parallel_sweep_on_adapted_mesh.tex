\documentclass[letterpaper]{article}
\usepackage{amsmath}
\usepackage{array}
\usepackage{color}
\usepackage{graphicx}
\usepackage{float} % utiliser H pour forcer a mettre l'image ou on veut
\usepackage{lscape} % utilisation du mode paysage
\usepackage{mathbbol} % permet d'avoir le vrai symbol pour les reels grace a mathbb
\usepackage{enumerate} % permet d'utiliser enumerate
\usepackage{moreverb} % permet d'utiliser verbatimtab : conservation la tabulation
\usepackage{stmaryrd} % permet d'utiliser \llbrackedt et \rrbracket : double crochet
\usepackage[noabbrev]{cleveref} % permet d'utiliser cref and Cref
\usepackage{caption} % permet d'utiliser subcaption
\usepackage{subcaption} % permet d'utiliser subfigure, subtable, etc
\usepackage[margin=1.in]{geometry} % controle les marges du document


\newcommand\bn{\boldsymbol{\nabla}}
\newcommand\bo{\boldsymbol{\Omega}}
\newcommand\br{\mathbf{r}}
\newcommand\la{\left\langle}
\newcommand\ra{\right\rangle}
\newcommand\bs{\boldsymbol}
\newcommand\red{\textcolor{red}}
\newcommand\ldb{\{\!\!\{}
\newcommand\rdb{\}\!\!\}}
\newcommand\llb{\llbracket}
\newcommand\rrb{\rrbracket}

\renewcommand{\(}{\left(}
\renewcommand{\)}{\right)}
\renewcommand{\[}{\left[}
\renewcommand{\]}{\right]}


\begin{document}
\title{Parallel Sweep on Adapted Mesh}
\author{Bruno Turcksin} 
\date{}
\maketitle

\section{Introduction}

\section{Results}
Uniform mesh $S_8$ (40 directions). $10\times10=100$, $30\times30=900$,
$50\times50=2500$, and $70\times70=4900$.
\begin{table}[H]
  \begin{center}
    \begin{tabular}{|c|c|c|c|c|}
      \hline
      Mesh size & Initial Number of Stages & Final Number of Stages& Optimal & Iterations \\
      \hline
      $10\times 10$ (no seed) &  580 &  56 &  56 & 1 \\
      $10\times 10$ (seed=0)  &  155 &  56 &  56 & 2 \\
      $10\times 10$ (seed=1)  &  153 &  56 &  56 & 1 \\
      $30\times 30$ (no seed) & 1780 &  96 &  96 & 1 \\
      $30\times 30$ (seed=0)  &  392 &  96 &  96 & 2 \\
      $30\times 30$ (seed=1)  &  376 &  96 &  96 & 2 \\
      $50\times 50$ (no seed) & 2980 & 136 & 136 & 1 \\
      $50\times 50$ (seed=0)  &  570 & 136 & 136 & 2 \\ % much faster than no
      % seed (less than one day instead of two)
      $50\times 50$ (seed=1)  &  573 & 136 & 136 & 2 \\
    \end{tabular}
  \end{center}
\end{table}


\end{document}

